% LaTeX template for MICS papers
% To Run:  pdflatex Sample.tex

\documentclass[12pt]{article}

\setlength{\oddsidemargin}{0in}
\setlength{\evensidemargin}{0in}
\setlength{\topmargin}{0in}
\setlength{\headheight}{0in}
\setlength{\headsep}{0in}
\setlength{\textwidth}{6in}
\setlength{\textheight}{9in}
\usepackage[parfill]{parskip}

\usepackage{graphicx} %For jpg figure inclusion
\usepackage{times} %For typeface

\begin{document}
\pagestyle{plain}

\title{Sample MICS Paper Title}

\author{
Mary Smith\\
Name of Department\\
Name of Institution\\
City, State Zip\\
Contact email
\and
Joe Jones\\
Name of Department\\
Name of Institution\\
City, State Zip\\
Contact email
}
\date{} 

\maketitle
\thispagestyle{empty}

\section*{\centering Abstract}

This is the abstract of the paper.  Note that the title and the abstract are the only things on the title page.  This example shows two authors at different institutions.  If information is the same for multiple authors, the authors can be listed as:

\begin{center}
{\large
Mary Smith and Joe Jones\\
Name of Department\\
Name of Institution\\
City, State Zip\\
Contact email
}
\end{center}  

\newpage
\setcounter{page}{1}

\section{Method} % (fold)
\label{sec:method}

\subsection{Algorithm} % (fold)
\label{sub:algorithm}
Our algorithm begins by reading in and expected amino acid order and the NMR carbon shift data from a data file. It then changes the amino acids to there expected $C_\alpha$ and $C_{\beta}$ values and stores them in an list that represents the protein chain. Next, the carbon shift values for a single amino acid is placed in a tile. A tile contains the $C_\alpha$ and $C_{\beta}$ values for the amino acid residue $i$ and residue $i-1$. Next, missing data is accounted for. If there are less tiles than the number of amino acids in the protein chain, place holder tiles are generated to make up the difference. Before the search for the best solution begins, the tiles are grouped by the type of amino acids they could represent to accelerate the assignment process. Tiles that do not match a group are grouped with the place holder tiles

To start the search, tiles in the place holder group and tiles in the same amino acid group as the first amino acid in the protein chain are placed in the first position in a node. The tile is then compared to the first amino acid in the protein chain, and a cost is generated. 

%continue assignment

%goal state

%solution found

% subsection algorithm (end)
% section method (end)
\bibliographystyle{amc}
\bibliography{Paper}

\end{document}
